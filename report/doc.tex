\documentclass[12pt, a4paper]{article}
\usepackage[margin=25mm]{geometry}
\usepackage[CJKnumber]{xeCJK}
\usepackage{ulem}
\usepackage{etoolbox}
\usepackage{paralist}
\usepackage{fontspec}
\setCJKmainfont[BoldFont={方正准雅宋_GBK}]{方正兰亭刊宋_GBK}
\setCJKfamilyfont{Caption}{方正粗雅宋_GBK}
\setCJKfamilyfont{SubCaption}{方正准雅宋_GBK}
\setCJKfamilyfont{Item}{方正标雅宋_GBK}
\setmonofont{Source Code Pro}
\usepackage{verbatim, minted, amsfonts, indentfirst}

\newcommand{\LabSrc}[1]{\inputminted[fontsize=\scriptsize, breaklines]{c}{./src/#1.c}}
\newcommand{\SetLabCol}[1]{\def\LabCol{#1}}
\newcommand{\AddLabCol}[1]{\uline{\makebox[12em][l]{#1}}}

\newcommand{\SetLabPerson}[2]{%
  \def\LabPersonName{#1}
  \def\LabPersonId{#2}
}

\newcommand{\AddLabItem}[2]{%
  \makebox[5em][r]{#1:} &
  \parbox[t]{12em}{#2}
  \vspace{1.5em}
  \\
}

\newcounter{LabYear}      \setcounter{LabYear}{\the\year}
\newcounter{LabYearStart} \setcounter{LabYearStart}{\the\year}
\newcounter{LabYearEnd}   \setcounter{LabYearEnd}{\the\year}
\newcounter{LabMonth}     \setcounter{LabMonth}{\the\month}
\newcounter{LabDay}       \setcounter{LabDay}{\the\day}

\newcommand{\SetLabDate}[3]{%
  \setcounter{LabYear}{#1}
  \setcounter{LabYearStart}{#1}
  \setcounter{LabYearEnd}{#1}
  \setcounter{LabMonth}{#2}
  \setcounter{LabDay}{#3}
}

\newcommand{\LabTitlePage}[3]{\begin{titlepage}
  \begin{center}
    \vspace*{1cm}
    \CJKfamily{Caption}     \Huge\textbf{北京邮电大学软件学院}  \\
    \CJKfamily{SubCaption}  \huge\textbf{%
      \ifnum\value{LabMonth} < 9%
        \def\LabSemester{二}
        \addtocounter{LabYearStart}{-1}
      \else
        \def\LabSemester{一}
        \addtocounter{LabYearEnd}{1}
      \fi
      \arabic{LabYearStart}--\arabic{LabYearEnd}学年第{\LabSemester}学期实验报告
    }

    \vspace{4em}

    \CJKfamily{Item}\large
    \begin{tabular}{ll}
      \AddLabItem{课程名称}{\uline{#1\hfill}}
      \AddLabItem{实验名称}{\uline{#2\hfill}}
      \AddLabItem{指导教师}{\uline{#3\hfill}}
      \AddLabItem{实验完成人}{%
        \ifundef{\LabCol}{\AddLabCol{\LabPersonName}}{\LabCol}
      }
      \vspace{1.5em}
    \end{tabular}

    \vspace{\fill}

    姓名:\underline{\,\LabPersonName\,}\,
    学号:\underline{\,\LabPersonId\,}\,
    成绩:\underline{\hspace{3em}}

    \vspace{2em}

    \begin{flushright}
      日期:\arabic{LabYear}年\arabic{LabMonth}月\arabic{LabDay}日
    \end{flushright}

    \vspace{2em}
  \end{center}

\end{titlepage}}

\begin{document}
\SetLabPerson{王明哲}{2014212080}
\LabTitlePage{操作系统}{内存分配器的实现}{刘知青}

\section{实验目的}
通过实现并优化自己的内存分配器,练习内存分配相关的数据结构和算法。使用 \texttt{mmap} 申请虚拟内存,通过 \texttt{LD\_PRELOAD} 在运行时替换系统库的调用。实现 C 的标准内存申请 API(\texttt{malloc, calloc, realloc, free})。

\section{实验环境}
  Arch Linux:\vspace{0.5em}
  \begin{compactitem}
    \item linux 4.5.4
    \item glibc 2.23
    \item make 4.1
    \item gcc 6.1.1
    \item gdb 7.11
  \end{compactitem}

\section{实验内容}
(整个项目和报告的源文件可以在 https://github.com/hghwng/toy-heap-manager 下载)

\subsection{综述}
使用基于 BiBOP 的分配策略:根据申请的记录大小,将分配请求分为两大类。
\begin{itemize}
  \item 当记录大于 1024 字节时,分配独立的页。
  \item 当记录小于 1024 字节时,将记录放置于包(bag)内。包存储着多个记录,还要维护包信息的元数据。一个包占用一整个页。包之间通过链表维护:
    \begin{itemize}
      \item 尚有剩余空间的包置于空链表中。若记录插入后,包不再有剩余空间,则移入满链表。若记录释放后,包内不再有有效记录时,释放整个包的空间。
      \item 已经存满数据的包置于满链表中。当有记录被释放时,移动当前包进入空链表。
    \end{itemize}
    以 $[2^i+1, 2^{i+1}], 3 \le i \le 10$ 将记录分类,将记录插入对应的包内。这样,可以确保任意时刻对于除最小包之外的所有包内记录空间的使用率不低于 50\%。
\end{itemize}

\subsection{数据结构}

\subsubsection{链表}
使用链表节点和有效数据分离的思路,双向环形链表的数据结构,通过宏实现基本的链表操作。

\LabSrc{list}

不同于一般的节点数据耦合的思路,遍历链表时,需要从给定的结构体成员的链表节点地址,恢复结构体本身的基地址。方法很简单,将结构体成员的地址直接减去结构体成员的偏移量即可。ANSI C 中提供了 \texttt{offsetof} 宏,使用 0 指针来获得节点相较于结构体的偏移量,不再赘述。

\LabSrc{list_em}

\subsubsection{Bitmap}
通过 bitmap 维护记录的存在性,每八个记录只需要一个字节,节省空间。在这里使用 64 bit 作为 bitmap 的基本单元(block)。使用 GCC 对于长度为 0 的数组的支持。

\LabSrc{bitmap}

除了基本的接口外,本项目需要高效地寻找可用记录,即在 bitmap 中获得某个被置位的比特的位置;还需要统计记录的使用情况,即在 bitmap 中对已置位的比特计数。

使用 gcc 提供的内置函数 \texttt{\_\_builtin\_popcount} 和 \texttt{\_\_builtin\_ctz} 可以在各个平台高效地进行比特计数和尾部 0 位置的寻找。结合基本单元的数据类型,使用 64 bit 的版本。注意到这两个方法对于开头的 0 不敏感,可以将末尾的非 64 整数倍的比特一并处理。

\LabSrc{bitmap_em}

\subsubsection{包头}
有了基本数据结构的支持,下面设计包头。为了提高内存访问效率,在一些平台下避免错误,需要对齐地址为处理器字长的整数倍。在目前测试的 AMD64 平台下,以 8 字节做对齐。

\LabSrc{align}

首先设计包的结构,并以 \texttt{bucket\_header} 命名。通过构造合理的包头结构,可以避免对齐带来的副作用。小的结构体成员尽量靠在一起,防止对齐导致空隙,降低内存的使用效率。最后面放置变长的 bitmap,而 bitmap 以 64bit 为基本单元,已经对齐,后面真实存放的用户数据就无需对齐了。

最后,由于 BLOB 和 bucket 两类数据不会在同一页中出现,使用 \texttt{union} 以重复使用相同的偏移量。

\LabSrc{bag}

对于大记录(BLOB)类型的数据,需要记录当前申请了多少页,在释放或重分配时会用到。对于小记录(bucket)类型的数据,需要放置双向环形链表节点,用来维护空包链表;需要存储当前的记录,在插入新记录时会用到。使用 \texttt{type} 域记录当前包的类型,并设置魔术值区分 BLOB 和 bucket:

\LabSrc{bag_def}

\subsection{初始化和内存申请}
在全局变量中存储 \texttt{/dev/zero} 的文件描述符和各个类型包的链表头。在动态库被载入时,需要初始化这些数据。依据 GCC 的文档,使用 \texttt{\_\_attribute\_\_} 将 \texttt{init} 函数置入 \texttt{.init\_array} 段,由链接器在程序初始化时调用。

\LabSrc{init}

使用 \texttt{mmap} 和 \texttt{munmap} 模拟对页的申请和释放:

\LabSrc{mem}

\subsection{大包管理}
注意到内存分配器在存储用户记录外,还要存储包头及对齐带来的空隙。以此计算需要的字节数并换算成页数。其余地方比较简单,不再赘述。


\LabSrc{blob}

\subsection{小包管理}
\subsubsection{计算}
由于小包根据记录大小分为不同的类别,所以不同类别的小包能存放的记录数是不同的。在分配空间时必须计算。

刨去上面提到的包头大小及对齐空隙,剩下的内容用来存放记录。每个记录实质上由 $2^i$ 字节的用户数据空间和 1 比特的可用状态 bitmap 组成。

\LabSrc{buck_calc}

在分配内存时,跳过包头和可变长度的 bitmap 才是用户数据区。在此区间进一步计算,可用获得当前记录所在用户数据区的偏移量。

在释放内存时,需要通过用户记录地址还原包头地址,以及当前记录在包中的位置。结合内存布局,仔细计算。计算过程比较繁琐,且涉及一定的 C 语言技巧,不再赘述。

\LabSrc{buck_addr}

\subsubsection{记录申请}
先分组。对于小于最小记录长度的,设置为最小记录长度。其余部分需要注意的是,一类包的适用空间是在 $[2^i+1, 2^{i+1}]$,在统计最高位位置时需要减去 1。直接计算最高位所在位置即可。确定分组后,要确保当前分组的存在空包。没有空包则需要申请。在获取空包后,通过 bitmap 的辅助,查找可用空记录在包中的位置。计算过程比较繁琐,不再赘述。最后,当空包变满后,需要移动当前包所在链表。

释放包的过程类似,设置 bitmap 后,检查状态。0 记录的包需要释放,否则移动进入空表链表。

\LabSrc{buck_alloc}

\subsection{外部接口}
对函数名称不进行 \texttt{static} 修饰,获得导出的符号,以便后期覆盖系统库的符号。

各类外部接口都需要对 \texttt{NULL} 地址和 0 长度的申请做特殊处理。

\texttt{malloc} 根据申请长度选择合适的包类型即可。

\texttt{calloc} 类似于 \texttt{malloc},但需要注意 \texttt{calloc} 将申请的空间全部置零。

\texttt{realloc} 先检查是否可以于原地重分配;若不可以,则调用 malloc 申请内存并拷贝,最后释放原始内存。

\texttt{free} 根据不同的包分类释放。

\LabSrc{interface}

\section{实验结果}

编写 Makefile,构建项目。

\inputminted[fontsize=\scriptsize, breaklines]{makefile}{../src/Makefile}

执行测试:

{\scriptsize
\begin{verbatim}
src git:(master) % make                                                          22:27
cc -c -g -Wall -Wextra -fPIC -DDEBUG=0 heap.c -o bin/heap.o
cc -shared -g bin/heap.o -o bin/libheap.so

src git:(master) % make run                                                      22:27
LD_PRELOAD=bin/libheap.so /bin/ls
bin  bitmap.h  heap.c  heap.h  list.h  Makefile


src git:(master) % make run APP='/bin/find .'                                    22:27
LD_PRELOAD=bin/libheap.so /bin/find .
.
./bin
./bin/heap.o
./bin/libheap.so
./bitmap.h
./heap.c
./heap.h
./list.h
./Makefile
./.heap.c.swp
./.heap.h.swp
./.list.h.swp
./.bitmap.h.swp
\end{verbatim}
}

进行较高强度的测试。在覆盖系统库符号的情况下,调用 \texttt{/bin/find /} 枚举全盘文件,调用 \texttt{/bin/bash} 执行脚本,调用 \texttt{/bin/gcc} 编译源文件,均未见执行异常。

\section{小结}
调试指针,调到泪奔!调位运算,忘记吃饭!


\end{document}
